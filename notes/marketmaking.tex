\documentclass[11pt,a4paper]{article}
\usepackage[utf8]{inputenc}
\usepackage{amsmath}
\usepackage{amsfonts}
\usepackage{amssymb}
\usepackage{listings}
\usepackage{xcolor}
\usepackage{hyperref}
\usepackage{geometry}
\usepackage{booktabs}
\usepackage{array}

\geometry{margin=1in}
\hypersetup{
    colorlinks=true,
    linkcolor=blue,
    filecolor=magenta,
    urlcolor=cyan,
}

\definecolor{codegray}{gray}{0.95}
\definecolor{codeblue}{rgb}{0.2,0.2,0.6}

\lstset{
    backgroundcolor=\color{codegray},
    basicstyle=\ttfamily\small,
    breaklines=true,
    captionpos=b,
    keepspaces=true,
    numbers=left,
    numbersep=5pt,
    showspaces=false,
    showstringspaces=false,
    showtabs=false,
    tabsize=2,
    frame=single,
    rulecolor=\color{black!30}
}

\title{Hyperliquid Market Making Study}
\author{}
\date{}

\begin{document}

\maketitle

\section{Executive Summary}

Hyperliquid implements a \textbf{purely algorithmic market making incentive system} based on transparent, volume-based negative maker fees. Unlike traditional exchanges with negotiated MM programs, Hyperliquid's system is permissionless and automatically scales with market share performance.

\section{Key Mechanisms}

\subsection{1. Maker Rebate System}
\begin{itemize}
    \item \textbf{Threshold}: 0.5\% of \textbf{maker volume} (not total volume)
    \item \textbf{Rebate Rate}: -0.001\% (-0.1 basis points)
    \item \textbf{Mechanism}: Automatic via \texttt{makerFractionCutoff} parameter in API
    \item \textbf{Payment}: Instant credit per trade, no claiming needed
\end{itemize}

\subsection{2. Base Fee Structure}
\begin{itemize}
    \item \textbf{Perpetuals}: Maker 0.015\%, Taker 0.045\%
    \item \textbf{Spot}: Maker 0.04\%, Taker 0.07\%
    \item \textbf{Spot volume counts 2x} toward VIP tiers
\end{itemize}

\subsection{3. Staking Discount System}
\begin{itemize}
    \item \textbf{Range}: 5\% to 40\% discount on base fees
    \item \textbf{Diamond Tier}: 40\% discount (500k+ HYPE staked)
    \item \textbf{Stackable}: Applied after volume tier reductions
\end{itemize}

\subsection{4. Volume-Tier Discounts}
\begin{itemize}
    \item \textbf{Tier 1}: \$5M 14-day volume \textrightarrow Maker 0.012\%, Taker 0.04\%
    \item \textbf{Progressive}: Higher tiers reduce both maker and taker fees
\end{itemize}

\section{Mathematical Framework}

\subsection{Effective Fee Calculation}
\begin{equation}
\text{effective\_maker\_rate} = \max\left(
\begin{array}{c}
\text{base\_maker\_rate} \times (1 - \text{volume\_discount}) \\
\times (1 - \text{staking\_discount}) \times (1 - \text{referral\_discount}) + \text{maker\_rebate},\\
\text{maker\_rebate} \quad \text{(Floor at rebate rate)}
\end{array}
\right)
\end{equation}

\subsection{Revenue Example (Diamond tier, 1\% maker share)}
\begin{lstlisting}[language=Python]
daily_volume = 1_000_000  # $1M daily volume
maker_rebate = 0.00001    # 0.001%
daily_revenue = daily_volume * maker_rebate  # $10 per $1M
annual_revenue = $3,650 per $1M daily volume
\end{lstlisting}

\section{API Integration}

\subsection{Key Endpoint}
\begin{lstlisting}[language=bash]
POST https://api.hyperliquid.xyz/info
Content-Type: application/json

{
  "type": "userFees",
  "user": "0x..."
}
\end{lstlisting}

\subsection{Response Structure}
\begin{lstlisting}[language=]
{
  "feeSchedule": {
    "add": "0.00015",           // Base maker fee
    "cross": "0.00045",         // Base taker fee
    "tiers": {
      "mm": [{                    // Market maker rebates
        "makerFractionCutoff": "0.005",
        "add": "-0.00001"
      }]
    }
  },
  "userAddRate": "0.000105",     // Your effective maker rate
  "userCrossRate": "0.000315",   // Your effective taker rate
  "activeStakingDiscount": {
    "discount": "0.3"            // 30% discount if applicable
  }
}
\end{lstlisting}

\section{Business Strategy}

\subsection{Competitive Advantages}
\begin{enumerate}
    \item \textbf{No collateral requirements} - capital efficient
    \item \textbf{Transparent mechanics} - all rates visible via API
    \item \textbf{Automatic scaling} - revenue grows with market share
    \item \textbf{Instant rebates} - no claiming or delays
\end{enumerate}

\subsection{Market Share Dynamics}
\begin{itemize}
    \item \textbf{\textless= 0.5\% maker share}: Pay reduced fees, compete on spread
    \item \textbf{\textgreater= 0.5\% maker share}: Earn rebates, can quote tighter
    \item \textbf{Network effects}: Larger MMs afford tighter spreads due to rebate income
\end{itemize}

\subsection{Break-Even Analysis}
For a market with \$500M daily volume:
\begin{itemize}
    \item \textbf{1\% maker share}: \$5M daily volume
    \item \textbf{Daily rebate}: \$50 (at 0.001\% rebate)
    \item \textbf{Annual revenue}: \$18,250 pure rebate income
    \item \textbf{Additional revenue}: Tighter spreads capture more flow
\end{itemize}

\section{Implementation Considerations}

\subsection{Risk Management}
\begin{itemize}
    \item \textbf{Inventory risk}: Must manage position exposure
    \item \textbf{Adverse selection}: Rebate doesn't protect against toxic flow
    \item \textbf{Competition}: 23 wallets currently qualify for enhanced rebates
\end{itemize}

\subsection{Technology Requirements}
\begin{itemize}
    \item \textbf{Low latency}: Need competitive execution to capture flow
    \item \textbf{Inventory management}: Dynamic position sizing
    \item \textbf{Market making algorithms}: Must optimize for both spreads and volume share
\end{itemize}

\section{Verification Status}

\subsection{Officially Confirmed}
\begin{itemize}
    \item Maker rebate mechanism via API documentation
    \item Base fee structure from official sources
    \item Staking discount tiers from 2025 implementation
    \item Mathematical formulas validated against API responses
\end{itemize}

\subsection{Cross-Referenced Sources}
\begin{enumerate}
    \item \textbf{Primary}: Hyperliquid GitBook API Documentation
    \item \textbf{Staking}: DeFi Planet (April 29, 2025 implementation)
    \item \textbf{Market data}: BlockBase Insights (23 qualifying wallets)
\end{enumerate}

\section{Conclusion}

Hyperliquid's market making system represents a \textbf{paradigm shift} from relationship-based to \textbf{performance-based liquidity provision}. The mathematical elegance lies in its simplicity: successful market makers are automatically rewarded through transparent, algorithmic mechanisms that create positive-sum dynamics between efficient MMs and the broader trading ecosystem.

The system is particularly attractive for \textbf{algorithmic traders} who can optimize for volume share rather than negotiating bilateral agreements, making it one of the most \textbf{capital-efficient market making opportunities} in the current landscape.

\end{document}